\chapter{Accessibility to technological and medical innovation}
\section{Introduction}
Over the past decades, technological innovation has played a crucial role in transforming our societies, influencing nearly every aspect of our daily lives, from communications to healthcare, education, the economy, and much more. As new technologies emerge rapidly, access to these innovations has become a significant concern to ensure an inclusive and equitable society.

Accessibility to innovation refers to the ability of all individuals, regardless of their social background, economic situation, place of residence, or physical condition, to benefit from and contribute to technological and scientific advancements.

Medical innovation has improved millions of people's health and quality of life worldwide. Revolutionary advances in healthcare, research, and technology have made diagnosing, treating, and preventing many previously considered incurable diseases possible. However, equitable access to these medical innovations remains a significant challenge for many populations, especially those in disadvantaged or marginalized regions.

Accessibility to medical innovation, therefore, refers to the capacity of individuals, communities, and countries to benefit from the latest medical advancements and derive tangible health benefits from them. This includes access to new treatments, medications, medical devices, and cutting-edge technologies in the healthcare field.

Unfortunately, many regions face significant challenges regarding access to medical innovation. Economic, social, and geographical disparities can lead to limited access to healthcare and innovative treatments. In developing countries, access to adequate healthcare infrastructure and advanced medical resources can be a significant hurdle for many individuals, especially in rural areas.

Furthermore, the high cost of innovative medical treatments and technologies can render these options unaffordable for many individuals, even in developed countries. This accentuates health inequalities and leaves vulnerable populations without access to the best medical solutions.

The challenge of accessibility to medical innovation requires a comprehensive and collaborative approach. Governments, international organizations, pharmaceutical companies, and healthcare stakeholders must work together to develop policies and strategies to remove financial, geographical, and structural barriers that hinder access to medical innovations.

\section{Related Work}
Accessibility to technological innovation aims to make technologies accessible to all individuals, regardless of their abilities or specific needs. This approach is based on the fundamental principle that advancements should be designed inclusively, allowing everyone to benefit from their opportunities. Here is an overview of the main aspects of accessibility to technological innovation:

Digital accessibility involves designing websites, mobile applications, software, and online content to be usable by all individuals, including those with visual, auditory, motor, or cognitive impairments. Researcher Shari Tewin has been involved in numerous projects based on assistive technologies, such as screen readers for blind people, adapted keyboards, eye-tracking interfaces, assistive communication devices, and more  [1-4]. Assistive technologies are devices and software designed to help individuals with specific needs use technology more effectively, improving digital accessibility.

The Internet of Things (IoT) refers to the process of connecting physical objects to the Internet. It opens up new opportunities for technological innovation, but it is essential to ensure that these connected devices are accessible to all users. This requires inclusive design and consideration of various needs. IoT is particularly relevant in the medical field with electronic health records, offering more precise, reliable, and accessible patient data. However, it can also pose privacy risks if misused [5]. IoT is increasingly present in connected greenhouses[6] and smart farming[7][8] areas, providing valuable support to farmers in monitoring and improving their product management.

Accessibility to technological innovation must also include ethical considerations for using artificial intelligence (AI), which is increasingly significant in modern technologies. It is essential to ensure that AI does not perpetuate existing biases or discrimination. AI solutions play a growing role in decision-making and interactions, potentially impacting positively and negatively. Considering the needs of users with disabilities can help technologists identify high-impact challenges whose solutions can advance AI for all users[9].

In some countries, laws and regulations have been implemented to promote accessibility to technological innovation. For example, the Americans with Disabilities Act (ADA) in the United States requires businesses and organizations to provide accessible services to people with disabilities, including online services. Medical technological innovation often involves collecting and analyzing large amounts of health data. Accessibility to these technologies must be balanced with significant ethical considerations regarding the protection of patient privacy and data confidentiality. Stringent security measures must be implemented to ensure that sensitive medical information is protected and used ethically[10]. In some cases, like in China, measures have been implemented to track and restrict the movements of its citizens during the COVID-19 pandemic, raising ethical concerns[11].

In conclusion, accessibility to technological innovation is an ever-evolving field that aims to ensure that technological advancements benefit everyone without exclusion. It’s an essential approach to building a more inclusive and equitable world where everyone can enjoy the opportunities and benefits of the ongoing technological revolution. The commitment of designers, policymakers, and society is necessary to continue progressing toward full accessibility in technological innovation.

\section{Ethics of Accessibility}
\subsection{Accessibility to Technological Innovation}

Ethics in technological accessibility is a crucial consideration in our modern society. As technology continues to advance rapidly, it is imperative to ensure that the benefits of this progress are accessible to all individuals, regardless of their specific needs or abilities. Technological accessibility encompasses the provision of hardware devices and the design and development of software and applications that allow everyone to access information and digital services equitably and inclusively.

As a society, we are responsible for ensuring that no one is left behind in this ever-expanding digital era. As Tim Berners-Lee, the creator of the World Wide Web, emphasizes, "The power of the Web is in its universality. Access by everyone, regardless of disability, is an essential aspect." [12].

Technological ethics also require recognizing the diversity of user needs. Each individual has different abilities and limitations, and it’s essential to design technological solutions that account for this variability.

This accessibility is based on the principles of equity and inclusion, ensuring that all individuals, regardless of constraints, can benefit from technology’s opportunities. As Vint Cerf, one of the pioneers of the Internet, reminds us: "An accessible Internet is an Internet for all."

Technology designers and developers must incorporate accessibility from the outset of the design process, as emphasized by Steve Ballmer, former CEO of Microsoft: "Accessibility is not a feature; it's a responsibility." This approach helps identify and address accessibility issues before they become obstacles for users. By focusing on accessibility, we acknowledge that every person has the right to access information, education, employment, and other essential services equitably. The ultimate goal of technological accessibility ethics is to build an inclusive world where everyone can fully participate and benefit from the ongoing technological revolution.

Technological accessibility extends beyond people with disabilities to individuals with specific needs based on age, culture, language, or socio-economic status. As highlighted by the United Nations in its report "Digital Inclusion for All: Empowering the Poor and Vulnerable" [13], technological accessibility plays a crucial role in reducing inequalities and empowering marginalized populations.

To achieve true technological accessibility, promoting collaboration is essential. Researchers have emphasized the importance of collaboration among designers, developers, users, and disability rights advocacy groups to ensure an inclusive, user-centered design. Active user participation throughout the development process is essential for identifying accessibility issues and finding suitable solutions, as noted by Shari Trewin in several articles [14][4].

Privacy protection and data security are also important aspects of technological accessibility ethics. As advanced technologies collect and analyze increasing amounts of data, it’s crucial to ensure that this information is used ethically and does not infringe on individuals' fundamental rights. Vital regulatory and ethical frameworks must be established to protect individuals' privacy while promoting innovation.

In conclusion, technological accessibility and its ethics are matters of social justice and respect for the fundamental rights of every individual. By ensuring technology's accessibility to all, we work towards a world where everyone can fully participate, contribute, and prosper. As Albert Einstein said: "The value of a man is in his ability to give and not in his ability to receive."
However, this challenge is compounded by the issue of accessibility to medical innovation, which represents a significant and growing part of innovation.

\subsection{Accessibility to medical Innovation}
The ethics of medical innovation accessibility is a fundamental topic sparking many debates in healthcare. As technological advancements continue transforming medicine and opening up new prospects for health, it’s crucial to ensure that these developments benefit the entire population, regardless of their socio-economic status or residence. This graphic representation highlights disparities in accessibility to prostheses in South Africa linked to origins, underscoring the challenges to be addressed [15].

%image 

According to an article published in "The Lancet" in 2018, "access to medical innovations remains unequal worldwide, with significant disparities between low-income and high-income countries" [16]. These disparities can be attributed to the high cost of advanced medical treatments and technologies, geographical and logistical barriers, and inadequate healthcare resources and infrastructure. For instance, Mali aimed to make healthcare accessible to all at its independence in 1960, but poverty and resource shortages prevented it from achieving its goals [17].

Medical innovation can offer disease diagnosis, treatment, and prevention solutions. However, these advancements can also be expensive, raising ethical questions about fairness and justice in access to healthcare. As the World Health Organization (WHO) reminds us: "The right to health includes access to essential healthcare services, medicines, and medical technologies for all, without discrimination." He estimates that 650 million people worldwide are disabled. This equates to approximately 10 pourcent of the world’s population. Of those people, 80 pourcent live in low-income countries [18][19]. It’s also estimated that while 35–40 million people currently require prosthetic or orthotic services, only 1 in 10 persons has access to such services [15]. Ethics of medical innovation accessibility demand a balance between promoting innovation and ensuring its benefits are not reserved for a privileged elite. Health policies and financing mechanisms must be in place to make these new technologies accessible to all individuals in developed or developing countries, whether they live in urban or rural areas. Several countries, like France, have started on this objective using calculation methods and analysis to ensure equitable distribution and accessibility throughout the territory [20].

One notable issue is related to patents. Unlike pharmaceutical manufacturers, many commentators argue that patents stifle biomedical research, for example, by preventing researchers from accessing patented materials or methods needed for their studies. Patents have also been accused of hindering medical care by increasing drug prices in poor countries [21]. The article's authors emphasized the need for policy and social initiatives to promote a more equitable distribution of medical advancements to improve global health as a whole.

Another critical dimension of medical accessibility ethics concerns equitable participation in clinical trials and research. A study by Yaqi Yuan at Wake Forest University, USA, found that in 30 countries, only 17 pourcent of people were satisfied with their healthcare facilities [22]. New therapies and medical technologies must be rigorously and ethically tested, and it is essential to include diverse and representative populations in these studies. This ensures that the results apply to various populations and that the benefits of innovation are distributed equitably. As some researchers highlight:

"Inclusive and diversified medical research is essential to ensure therapies and treatments are suitable for all patients."

Digital technologies and mobile health applications offer considerable potential to improve medical innovation accessibility. A research report from the World Health Organization (WHO) in 2019 stated that "digital health technologies can play a key role in improving access to healthcare in remote areas and low-income countries" [23]. This can include solutions such as telemedicine, chronic disease tracking apps, and health education tools. However, this raises ethical concerns about data privacy and health data protection. With the advent of information and communication technologies in healthcare, it is essential to ensure the security and privacy of patient medical information. As American cardiologist and researcher Eric Topol pointed out:

"Future medical technologies can only succeed if they preserve patient privacy."

This implies establishing strict data protection standards and ensuring that access to medical information is restricted to authorized healthcare professionals.

The COVID-19 pandemic has exposed weaknesses in global healthcare, highlighting challenges in international coordination, information exchange, and healthcare accessibility. Healthcare systems have suffered from a lack of preparedness and coordination, exacerbated by disparities in medical resources. The rapid exchange of medical information has proven crucial, underscoring the need for a global platform to facilitate this communication. Furthermore, unequal access to healthcare has heightened the urgency of making medical innovations accessible to everyone, regardless of their resources [24].

In conclusion, the ethics of medical innovation accessibility is a significant concern in our quest to improve healthcare and address current and future medical challenges. By focusing on equity, inclusion, and the protection of patient rights, we can ensure that the benefits of medical advancements are extended to all, regardless of socio-economic or geographical context. Society and policymakers are responsible for ensuring that medical innovation is accessible to all, thus contributing to a more ethical, sustainable, and well-being-centered healthcare system. Many areas touch upon medical innovation accessibility. Although not well-known to the general public, the field of prosthetics represents a significant issue with unique complexities.

\subsection{Accessibility to The Prosthetics Field}
The ethics of prosthetic accessibility is a crucial issue that raises significant ethical and social considerations. Prosthetics are vital in improving the quality of life for amputees and individuals with physical disabilities, allowing them to regain lost mobility and independence. However, accessibility to these technologies raises questions about equity, costs, quality, and patient rights.

Prosthetic accessibility primarily involves two aspects: physical access and financial access. Physical access refers to the availability of prosthetics tailored to the specific needs of individuals, while financial access pertains to people's ability to afford prosthetics at a reasonable cost. In low-income or middle-income countries, limited prosthetic access can be due to financial constraints and limited healthcare services. This is the case in Sierra Leone, where access is minimal, with limited staff, leading to amputees being isolated from the population. This gap is exacerbated between rural and urban areas [25].

In the USA, annual prosthetic service caps in private health care plans typically range from \$500 to \$3000 in annual coverage. Lifetime restrictions have an even greater range, with some insurance plans covering up to \$10,000 and others only a single device during an amputee’s lifetime [26]. These caps limit access to prostheses, particularly to high-tech devices, which are considerably more expensive, showing inequality even within countries with higher access [27]. A high budget does not guarantee quality; for instance, countries like the USA spend several billion dollars on their healthcare system but face issues with quality and price. In contrast, countries like Costa Rica, Thailand, and Singapore spend an average amount but provide high-quality healthcare. Thus, it is crucial to focus on quality and healthcare accessibility for countries [28].

The United Nations Convention on the Rights of Persons with Disabilities emphasizes the importance of ensuring access to assistive technologies, including prosthetics, for disabled individuals:

"States Parties commit to ensuring and promoting the effective access of persons with disabilities to new technologies and information and communication systems, including the Internet." Notably, countries like France and international organizations like the WHO are working to promote innovation and make it accessible to all through action plans or agreements [20][29][30][31].

Many organizations, including nonprofits and hospitals, are attempting to improve accessibility to prosthetic devices. Despite the attempts of these various organizations, 95 pourcent of the amputee population in developing countries still lacks access to proper prosthetic care and affordable devices. Prosthetic accessibility extends beyond providing physical devices and encompasses access to specialized medical care, training for their use, and ongoing support to meet individual needs. In Kenya, the Orthopedic Technology Department of the National Hospital employs an average of 18 people to serve around 280 patients each month. The annual budget of this hospital is \$8,000 [32], highlighting the limits of medical innovation accessibility in poor countries.

Within the framework of the ethics of prosthetic accessibility, it is crucial to ensure that these technologies are available to everyone, regardless of their ability to pay. Efforts must be made to make prosthetics affordable and accessible to people from all walks of life. The World Health Organization estimates that, in the developing world, there are 40 million amputees, and only 5 pourcent of them have access to any form of prosthetic care [33]. Research conducted by Gulrez [34] underscores the importance of user-centered design in prosthetic development. These studies emphasize that considering users' needs and preferences is essential for improving the acceptance and effectiveness of prosthetics.

Access to advanced technologies is another aspect of the ethics of prosthetic accessibility. While new technological advancements continuously enhance prosthetics, it’s essential to ensure that these technologies are not only available to a privileged elite but benefit all those in need.
Finally, the confidentiality and security of user data with prosthetics must be considered. As mentioned earlier in the section on medical innovation accessibility, it’s essential to ensure that the sensitive medical data of prosthetic users is protected and that its use complies with ethical and legal standards.

In conclusion, the ethics of prosthetic accessibility is a complex issue that requires deep reflection on equity, financial access, user-centered design, and data security. By ensuring fair and affordable access to prosthetics, developing them with user needs, and protecting their privacy, we can improve the lives of millions worldwide and promote a more inclusive and equitable society.

\section{Project impact}
This project represents an initiative aimed at promoting accessibility and innovation in the field of prosthetics by integrating the principles of haptics and 3D manufacturing. The goal of this work is to restore sensations in amputated individuals by creating a kit designed for forearm prosthetics. It opens the way for a thought process based on accessibility to innovation, particularly in the still relatively unexplored areas of haptics and prosthetics. While these areas remain unfamiliar to the general public, this project serves as a first step to raise awareness and expand access to these technologies.

\textit{Prosthetics}

Accessibility to prosthetics has profound societal implications. While some are privileged to access sophisticated and custom prosthetics, many face financial and geographical barriers. As a result, progressive initiatives are emerging to rebalance this disparity by providing accessible and functional prosthetics. This project presents a simplified initial approach to acquiring a device to enhance prosthetics. It allows individuals to actively engage in rehabilitation by involving them in the design and improvement process. The project aims to remain simple while providing all the necessary resources to enable as many people as possible to adopt this approach. Currently, modeling software is limited to a minority of the population, as is the knowledge of 3D printing materials and their use in prosthetics. The goal is to ensure that every amputated individual, regardless of background, can benefit from prosthetics that improve their quality of life, mobility, and well-being.

\textit{Haptics}

Although it represents a crucial technological advancement, Haptics remains a relatively unknown field to the general public. Electronic devices open the door to rich sensory experiences, but technical and financial barriers limit their adoption. Efforts are being made today to democratize access to haptics. This includes the design of affordable devices, creating user-friendly software, and raising awareness of the possibilities offered by this technology. Accessibility to haptics significantly impacts various fields, such as medicine and virtual reality. This project takes an approach that allows individuals with no prior experience in haptics to use this technology to improve their prosthetics, among other applications. It assists users in their initial steps into haptics while remaining simple and accessible to all. By introducing the field and its various components and suggesting various possible uses, the project represents an affordable and easily accessible technological advancement. However, its primary goal remains the improvement of prosthetics.

This project, though preliminary, illustrates the intersection of technological innovation and social accessibility. By providing an affordable and user-friendly kit, offering components and code accessible to all, and 3D prosthetic prototypes, this initiative aims to strengthen the ethics of technological accessibility. The combination of haptics and prosthetic concepts makes this approach innovative and potentially replicable on a larger scale.

In the end, this project represents a first step toward inclusive technological accessibility. By integrating technological advances accessible to all, it addresses inequalities in access to innovation. By combining haptic innovation with prosthetic design, this project is based on a thought model centered on accessibility to innovation, which is now a crucial step for the future of our society, thus improving the lives of individuals and contributing to the transformation of a more equitable society.

\section{Conclusion}
The accessibility of technological and medical innovation represents a significant challenge of the 21st century, transcending economic, societal, and political borders. This issue embodies an unprecedented opportunity to steer society toward more significant equity, progress, and inclusion. At the intersection of political, economic, and societal issues, equitable access to technological and medical advancements drives global impact and transformation.

From a political and economic standpoint, innovation is central to global competitiveness. Nations that prioritize innovation as a strategic cornerstone shape the international landscape by generating innovative ideas and influencing global standards. This dynamic brings global recognition to countries and bestows a robust and influential voice in global negotiations. Enhanced accessibility to innovation enables a country to engage in international trade and contribute to global exchanges.

From a societal perspective, universal accessibility to innovation fosters regional and national attractiveness. Regions open to innovation become incubators of economic growth and social well-being. By creating an environment conducive to innovation, these areas attract talent, stimulate foreign investments, and infuse creative energy. Consequently, accessibility to technological and medical innovation serves as an engine for development, reinforcing social cohesion and the global appeal of nations.

Key stakeholders, whether governments, international organizations, or businesses, play a pivotal role in ensuring accessibility to innovation. They are responsible for designing and implementing policies and initiatives that balance technological development and social inclusion. However, the impact of local initiatives and smaller-scale actors should not be underestimated. These initiatives have the power to catalyze research and initiate significant changes on a small scale, often leading to far-reaching repercussions beyond their original scope.

In summary, accessibility to technological and medical innovation is an opportunity to reevaluate the foundations of our contemporary societies. Political, economic, and societal considerations underscore the imperative to open the doors of innovation to all, without distinction, by promoting equity and inclusion. Whether through global or local initiatives, accessibility to innovation offers a future where technological progress knows no bounds, and each individual can contribute to collective flourishing.