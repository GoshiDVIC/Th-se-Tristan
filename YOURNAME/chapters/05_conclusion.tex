\chapter{Conclusion}

This thesis contributes to the understanding and development of haptics for medical innovation, specifically in the field of prosthetics, intended for both experienced and novice users through a DIY kit.

Understanding the role of technology and its accessibility is crucial in today's society and that of tomorrow. It is a significant step in realizing all new projects and must be considered by researchers and businesses. In this first chapter, an analysis and questioning of technological accessibility at various scales are carried out. Starting with technological innovation as a whole, then medical innovation, and finally prosthetics. This thesis provides a comprehensive reflection and encourages the approach to making technology more accessible while linking it to the thesis project.

The second chapter focuses on the field of haptics and takes a first step toward it by exploring various components and their possible uses. It explores different functions and provides simple codes for various uses. Through multiple prototyping and various tests, it offers solutions for implementing different components used for haptics that can be placed on a prosthesis.

The final chapter explores the development of a user-friendly and accessible prosthesis for everyone. This is achieved through selecting materials, a design method, and the design itself. The chapter addresses how to design to cater to a wider audience, how to implement different components into the prosthesis, and how to ensure longevity through clever design and material choices. The second section of this chapter deals with the creation of bio-patches made from accessible and user-friendly bio-materials. Prototyping and material and user tests are included in this thesis.

This thesis may be limited to individuals already present in these various fields. However, there are numerous possibilities for improvement with the constant evolution of medical innovation and the emerging field of haptics. This first step can be pushed further by creating increasingly advanced kits with new components, code improvement and optimization, and prosthetics designed for such devices.

This thesis is a first step toward haptics and medical innovation in prosthetics by offering a simple DIY kit accessible to everyone. It combines programming, understanding of haptics, 3D design, and bio-materials. Furthermore, this work presents methodological contributions to the integration of haptics into the field of prosthetics. It is accompanied by prototyping with methods to combine innovation, efficiency, and accessibility.