\chapter{First Step into the Field of Haptic}
\section{Introduction}

\section{Related Work}

\section{Evaluation}

\section{Limitations}
It’s essential to note that this chapter serves as an introduction and may not necessarily represent a significant contribution for individuals with advanced expertise in this field. It aims to remain accessible to a broad audience, whether in innovation and technology, design and engineering, or even economic aspects. That's why it delves deeply into haptic to ensure clear understanding while keeping it simple.

\textit{Hardware}

Currently, the prototype uses only one sensor connected to the micro-controller. The next phase, which involves placing a sensor on each finger of the prosthesis, could pose constraints in power, data reading capacity, and data output. The DRV2605 haptic motor driver controls only one ERM or LRA at a time. The overall size of the components can be excessive, limiting its use for certain prostheses and users. Additionally, the motors used are currently too bulky. The haptic effect becomes much less interesting by reducing the mass and increasing the frequency.

\textit{software}

The current code continuously reads data in real time, which keeps it in constant operation. The accumulation of data and some noise can slow down the entire system.

\section{Future Works}
Innovation in Haptic is still relatively new and represents a path for improvement in engineering. As mentioned earlier, the work presented in this chapter marks a first step and opens the way for numerous potential enhancements.

From a hardware perspective, one improvement for this project is creating a printed circuit board (PCB) to reduce the space occupied by all the components. This is crucial to ensure the efficient use of prosthetics, ensuring that the entire setup is optimal in terms of distance without causing interference or adding extra constraints to the user. Given the rapid evolution in Haptic, it's also essential to stay updated on new components to determine if they might be better suited and more efficient for this project. Using a controller capable of driving multiple DRV2605 devices is worth considering.

There are opportunities to enhance the project on the software front by integrating new features related to the components into the code. By optimizing the code for faster and more efficient operation, it becomes possible to improve the precision of vibrations. Other features associated with the controllers can be explored to optimize the code and offer new functionalities.

\section{Conclusion}
This chapter represents a first step towards understanding and accessing haptic technology. It explores various types of components, their functionalities, and different ways to use them. Furthermore, it offers multiple practical applications and explicit code that everyone can use. It aims to remain simple and accessible to all. The next chapter delves into prosthetics and presents different prototypes for integrating the haptic components introduced in this chapter.