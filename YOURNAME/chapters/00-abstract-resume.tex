Today, technology plays an increasingly important role at the heart of global issues; technological accessibility has become imperative for improving the quality of life for everyone, including individuals with specific needs. This thesis explores the intersection between haptic technology, prosthetics, and technological accessibility. 

The development of a DIY (Do It Yourself) prosthesis kit is at the heart of this research, offering an innovative approach to empower users to design and customize their prosthetics. By harnessing the capabilities of haptic technology, this prosthesis kit aims to enhance user experiences by providing precise and intuitive sensory feedback. 

Throughout this thesis, we will address several important steps, including prosthetic design, selecting suitable materials, and integrating haptic components to enhance sensory perception. Additionally, we will explore the possibilities bio-materials offer to create bio-patches integrated into prosthetics, thereby opening new horizons in rehabilitation.

This research advocates for a straightforward approach by emphasizing the convergence of technology and accessibility. Furthermore, it paves the way for democratizing medical innovation by enabling users to take charge of their rehabilitation.

Beyond the technical results, this thesis contributes to a reflection on how technology can serve inclusion and improve quality of life, strengthening our understanding of the relationship between humans and machines in the context of modern prosthetics.
