%\newcommand{\bmmax}{2}
%\newcommand{\hmmax}{0}
  
\usepackage[english]{babel}
\usepackage[english=british]{csquotes}

\usepackage{blindtext}
\usepackage{graphicx}

\usepackage{titletoc}


\usepackage{svg}
\usepackage{algorithmic}

\definecolor{webbrown}{rgb}{.6,0,0}

%% CHANGE CITE COMMAND
\renewcommand{\cite}[1]{%
~\citep{#1}%
}

%%%%%%%%%%%%%%%%%%%%%%%%%%%%%%%%%%%%%%%%%%%%%%%%%%%%%%%%%%
%%% INCLUSION / EXCLUSION %%%%%%%%%%%%%%%%%%%
\usepackage{microtype}
\usepackage{comment}
% !!! Comment or uncomment line under to exclude or include the content of the chapter:
%\excludecomment{content} % exclude the content, (only get introduction and summary)
\includecomment{content} % include the content, (get eveevolutionrything)
\includecomment{export}
%%%%%%%%%%%%%%%%%%%%%%%%%%%%%%%%%%%%%%%%%%%%%%%%%%%%%%%%%%


%%
% For nicely typeset tabular material
\usepackage{booktabs}
%%
% For graphics / images
\usepackage{graphicx}
\setkeys{Gin}{width=\linewidth,totalheight=\textheight,keepaspectratio}
\graphicspath{{graphics/}}
% The fancyvrb package lets us customize the formatting of verbatim environments.  We use a slightly smaller font.
\usepackage{fancyvrb}
\fvset{fontsize=\normalsize}


\usepackage{xspace}





% Prints an epigraph and speaker in sans serif, all-caps type.
\newcommand{\openepigraph}[2]{%
  %\sffamily\fontsize{14}{16}\selectfont
  \begin{fullwidth}
  \sffamily\large
  \begin{doublespace}
  \noindent\allcaps{#1}\\% epigraph
  \noindent\allcaps{#2}% author
  \end{doublespace}
  \end{fullwidth}
}

% Inserts a blank page
% \newcommand{\blankpage}{\newpage\hbox{}\thispagestyle{empty}\newpage}

\usepackage{units}

% Typesets the font size, leading, and measure in the form of 10/12x26 pc.
\newcommand{\measure}[3]{#1/#2$\times$\unit[#3]{pc}}

% Macros for typesetting the documentation
\newcommand{\hlred}[1]{\textcolor{Green}{#1}}% prints in red
\newcommand{\hangleft}[1]{\makebox[0pt][r]{#1}}
% \newcommand{\hairsp}{\hspace{1pt}}% hair space
\newcommand{\hquad}{\hskip0.5em\relax}% half quad space
\newcommand{\TODO}{\textcolor{red}{\bf TODO!}\xspace}
% \newcommand{\ie}{\textit{i.\hairsp{}e.}\xspace}
% \newcommand{\eg}{\textit{e.\hairsp{}g.}\xspace}
% \newcommand{\na}{\quad--}% used in tables for N/A cells
\providecommand{\XeLaTeX}{X\lower.5ex\hbox{\kern-0.15em\reflectbox{E}}\kern-0.1em\LaTeX}
\newcommand{\tXeLaTeX}{\XeLaTeX\index{XeLaTeX@\protect\XeLaTeX}}
% \index{\texttt{\textbackslash xyz}@\hangleft{\texttt{\textbackslash}}\texttt{xyz}}
\newcommand{\tuftebs}{\symbol{'134}}% a backslash in tt type in OT1/T1
\newcommand{\doccmdnoindex}[2][]{\texttt{\tuftebs#2}}% command name -- adds backslash automatically (and doesn't add cmd to the index)
\newcommand{\doccmddef}[2][]{%
  \hlred{\texttt{\tuftebs#2}}\label{cmd:#2}%
  \ifthenelse{\isempty{#1}}%
    {% add the command to the index
      \index{#2 command@\protect\hangleft{\texttt{\tuftebs}}\texttt{#2}}% command name
    }%
    {% add the command and package to the index
      \index{#2 command@\protect\hangleft{\texttt{\tuftebs}}\texttt{#2} (\texttt{#1} package)}% command name
      \index{#1 package@\texttt{#1} package}\index{packages!#1@\texttt{#1}}% package name
    }%
}% command name -- adds backslash automatically
\newcommand{\doccmd}[2][]{%
  \texttt{\tuftebs#2}%
  \ifthenelse{\isempty{#1}}%
    {% add the command to the index
      \index{#2 command@\protect\hangleft{\texttt{\tuftebs}}\texttt{#2}}% command name
    }%
    {% add the command and package to the index
      \index{#2 command@\protect\hangleft{\texttt{\tuftebs}}\texttt{#2} (\texttt{#1} package)}% command name
      \index{#1 package@\texttt{#1} package}\index{packages!#1@\texttt{#1}}% package name
    }%
}% command name -- adds backslash automatically
\newcommand{\docopt}[1]{\ensuremath{\langle}\textrm{\textit{#1}}\ensuremath{\rangle}}% optional command argument
\newcommand{\docarg}[1]{\textrm{\textit{#1}}}% (required) command argument
\newenvironment{docspec}{\begin{quotation}\ttfamily\parskip0pt\parindent0pt\ignorespaces}{\end{quotation}}% command specification environment
\newcommand{\docenv}[1]{\texttt{#1}\index{#1 environment@\texttt{#1} environment}\index{environments!#1@\texttt{#1}}}% environment name
\newcommand{\docenvdef}[1]{\hlred{\texttt{#1}}\label{env:#1}\index{#1 environment@\texttt{#1} environment}\index{environments!#1@\texttt{#1}}}% environment name
\newcommand{\docpkg}[1]{\texttt{#1}\index{#1 package@\texttt{#1} package}\index{packages!#1@\texttt{#1}}}% package name
\newcommand{\doccls}[1]{\texttt{#1}}% document class name
\newcommand{\docclsopt}[1]{\texttt{#1}\index{#1 class option@\texttt{#1} class option}\index{class options!#1@\texttt{#1}}}% document class option name
\newcommand{\docclsoptdef}[1]{\hlred{\texttt{#1}}\label{clsopt:#1}\index{#1 class option@\texttt{#1} class option}\index{class options!#1@\texttt{#1}}}% document class option name defined
\newcommand{\docmsg}[2]{\bigskip\begin{fullwidth}\noindent\ttfamily#1\end{fullwidth}\medskip\par\noindent#2}
\newcommand{\docfilehook}[2]{\texttt{#1}\index{file hooks!#2}\index{#1@\texttt{#1}}}
\newcommand{\doccounter}[1]{\texttt{#1}\index{#1 counter@\texttt{#1} counter}}




%\geometry{textwidth=.55\paperwidth}


% Generates the index
\usepackage{makeidx}
\makeindex

%%%%
\makeatletter
\renewcommand*\l@figure{\@dottedtocline{1}{1.5em}{2.3em}}
\makeatother

%% change TOC
%\setcounter{tocdepth}{2}
\setcounter{secnumdepth}{2}

\usepackage{pifont}% http://ctan.org/pkg/pifont
\usepackage{graphicx}
\usepackage{multirow}
\usepackage{xspace}
\usepackage{tabularx}
\usepackage{color}
\usepackage{listings}
\usepackage{colortbl}
\usepackage{morefloats}
\usepackage{enumitem}
\usepackage{rotating}
\usepackage{comment}
\usepackage{rotating}
% \usepackage[sort, numbers]{natbib} 
\usepackage[retainorgcmds]{IEEEtrantools}
\usepackage{bibentry}
\usepackage{longtable}
\usepackage{glossaries}
\usepackage{gensymb}
\usepackage{csvsimple}
\usepackage{amsmath}
\usepackage{cleveref}% Has to be loaded after hyperref
\usepackage[utf8]{inputenc}
\usepackage{todonotes}
\usepackage{marginfix}
\usepackage[export]{adjustbox}
\usepackage{fullwidth}

\usepackage[strict]{changepage}

\setlist[description]{style = multiline, labelwidth = 55pt}
\usepackage[parfill]{parskip}
\makeatletter
\makeatother

\usepackage{tikz,lipsum}
\usepackage[most]{tcolorbox}

\tcbset{titre/.style={boxed title style={boxrule=0pt,colframe=white}}}

\definecolor{gradientGreenL}{HTML}{1fe2ad} 
\definecolor{gradientGreenR}{HTML}{d4eb6f} 


\newtcolorbox{BoxResume}[2][]{
                boxrule=0.5pt,
                colback=white,
                top=3pt,bottom=2pt,left=2pt,right=2pt,
                colframe=webbrown,
                fonttitle=\sffamily\small,%\bfseries
                coltitle=black,
                colbacktitle=white,
                enhanced,
                attach boxed title to top left={xshift=5mm, yshift=-2mm},
                title=#2,#1
                }


%\newtcolorbox{BoxIn}{
%enhanced,
%colframe=white,
%interior style={
%left color=gradientGreenL!7!white,
%right color=gradientGreenR!7!white},
%%frame style image=background\aa.jpg
%left=5mm,
%top=4mm,
%bottom=4mm,
%right=5mm,
%boxsep=0mm,
%nobeforeafter}
%
%
%
%\newtcolorbox{BoxResumeNew}[2][]{
%                boxrule=1pt,
%                colback=white,
%                top=3pt,bottom=2pt,left=2pt,right=2pt,
%                colframe=black,
%                fonttitle=\sffamily\small,%\bfseries
%                coltitle=black,
%                colbacktitle=white,
%                enhanced,
%                attach boxed title to top left={xshift=5mm, yshift=-2mm},
%                title=#2,#1
%                }
%
%
%\newtcolorbox{BoxInNew}{
%enhanced,
%colframe=white,
%colback=black!2!white,
%%frame style image=background\aa.jpg
%left=5mm,
%top=4mm,
%bottom=4mm,
%right=5mm,
%boxsep=0mm,
%nobeforeafter}


%
%\newcommand{\remember}[1]{
%\vspace*{\fill}
%\begin{BoxResumeNew}[titre]{WHAT YOU MUST REMEMBER}
% \begin{BoxInNew}{}
% #1
% \end{BoxInNew}{}
%\end{BoxResumeNew}
%\vspace{0.5cm}
%} 


%%%%%%%%%%%%%%%%%%%%%%%%%%%%%%
%%%%%%%%%   QUOTE %%%%%%%%%%%%%%%%%%%
%%%%%%%%%%%%%%%%%%

\makeatletter
\renewcommand{\@chapapp}{}% Not necessary...
\newenvironment{chapquote}[2][2em]
  {\setlength{\@tempdima}{#1}%
   \def\chapquote@author{#2}%
   \parshape 1 \@tempdima \dimexpr\textwidth-2\@tempdima\relax%
   }
  {\par\normalfont\hfill--\ \chapquote@author\hspace*{\@tempdima}\par\bigskip}
\makeatother

\usepackage{styles/kaobiblio}
%\usepackage{styles/mdftheorems}

\usepackage{subfigure}
\usepackage{wrapfig}
\usepackage{float}

\usepackage{siunitx}

\usepackage{ifthen}
\usepackage{clipboard} %txa~apologies in advance :P
\newboolean{grayscale}     
\setboolean{grayscale}{false}% for the colored version   

\usepackage{minted}
\usemintedstyle{tango}


\newcommand{\CIRCLE}{$\bullet$}
\newcommand{\Circle}{$\circ$}


\graphicspath{{images/}}

\makeindex[columns=3, title=Alphabetical Index, intoc] 
\makeglossaries
\makenomenclature 
\makeglossaries

\setglossarystyle{listgroup}



\renewcommand{\thefootnote}{\arabic{footnote}}
\newcommand{\commentaire}[2]{\textcolor{#1}{(#2)}} 
\newcommand{\bp}[1]{\textcolor{cyan}{[B: #1]}} 
\newcommand{\mn}[1]{\textcolor[RGB]{100,100,200}{[M: #1]}}
\newcommand{\tbd}[1]{\textcolor[RGB]{255,0,0}{[#1]}}
\newcommand*\rot{\rotatebox{90}} %for rotated table headers
\newcommand{\cmark}{\ding{51}}% loading symbols from pifont
\newcommand{\xmark}{\ding{55}}% % loading symbols from pifont
\newcommand\tikzmark[1]{%
  \tikz[remember picture,overlay] \node (#1) {};}
  
\newacronym[longplural={Frames per Second}]{fpsLabel}{FPS}{Frame per Second}
\newacronym[longplural={Tables of Contents}]{tocLabel}{TOC}{Table of Contents}


\newacronym{DVIC}{DVIC}{De Vinci Innovation Center}
%---------------------------------------------------


% Only part in Table of Content
\setcounter{tocdepth}{-1}
\addto\extrasenglish{\renewcommand{\partname}[1]{}}
\renewcommand{\thepart}{}

\makeatletter
\@addtoreset{chapter}{part}
\makeatother

\makeatletter
\renewcommand\tableofcontents{%
  \null\hfill\Huge\textbf{List of Theses}\hfill\null\par
  \@mkboth{List of Thesis}{List of Thesis}%
  \@starttoc{toc}%
}
\makeatother


\newenvironment{thesis}[5]{
    \pagelayout{wide}
    \startcontents[partialTOC] 
    \part[#1]{    
            #1 \\ 
            %\vspace{0.5cm}
            \begin{center}
                \begin{minipage}[c]{9cm}
                   \centering
                   \large
                   #2
                \end{minipage}
                \\
                \vspace{1cm}
                \begin{minipage}[l]{13cm}
                    \normalsize
                    \normalfont
                    \rmfamily
                    \input{#3}
                \end{minipage}
            \end{center}
        }
	%\vspace{1cm}
	\addtocontents{ptc}{\begingroup}
	\addtocontents{toc}{
		\vspace{0.3cm}
		\hbox{
			\hspace{2cm}
			\begin{minipage}{2.5cm}
				\includegraphics[width=2.5cm]{#4}
			\end{minipage}
			\hspace{0.5cm}
			\begin{minipage}{9cm}
				\small \textbf{#2} - #5 
			\end{minipage}
			}
		\vspace{-1.7cm}
		}
	\addtocontents{ptc}{\endgroup} 
    \section*{Contents}
    \noindent\makebox[\linewidth]{\rule{\textwidth}{0.4pt}}
    \begin{spacing}{0.1}
    	\printcontents[partialTOC]{}{0}{\setcounter{tocdepth}{2}}
    \end{spacing}
    \pagelayout{margin}
    \cleardoublepage\bigskip
    }
    {
   \stopcontents[partialTOC]
    }
